\documentclass{article}
\usepackage{graphicx}			%paquete para importar imagenes
\usepackage{subfig}				%referenciar imagenes y ponerlas en serie
\begin{document}
MAYA SORAYDE
\begin{figure}[h]		%poner la imagen en: h->donde esta el codigo;t->arriba;b->abajo
\centering
\includegraphics[width=0.50\linewidth]{MayaSorayde.jpg}
\caption{Maya Sorayde}			%nombre imagen
\label{fig:Maya}				%etiqueta de la imagen
\end{figure}
BOXXY
\begin{figure}[h]		%poner la imagen en: h->donde esta el codigo;t->arriba;b->abajo
\centering
\includegraphics[width=3cm,height=2cm]{boxxy.png}
\caption{Boxxy}			%nombre imagen
\label{fig:Boxxy}				%etiqueta de la imagen
\end{figure}

\begin{figure}[h]		%poner la imagen en: h->donde esta el codigo;t->arriba;b->abajo
\centering
\subfloat[primera]{
\includegraphics[width=5cm]{MayaSorayde.jpg}
}
\hspace{0.1\linewidth}
\subfloat[segunda]{
\includegraphics[width=5cm]{MayaSorayde.jpg}
}
%es necesario poner este espacio para que esten alineados 2 arriba y 2 abajo.
%funciona tambien con \includegraphics[width=0.50\linewidth]{MayaSorayde.jpg}
\subfloat[tercera]{
\includegraphics[width=5cm]{MayaSorayde.jpg}
}
\hspace{0.1\linewidth}
\subfloat[cuarta]{
\includegraphics[width=5cm]{MayaSorayde.jpg}
}
\caption{Maya Sorayde}			%nombre imagen
\label{fig:Maya}				%etiqueta de la imagen
\end{figure}






\end{document}